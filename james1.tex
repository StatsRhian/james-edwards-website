 \documentclass[a4paper,10pt]{article}

\usepackage{url}
\usepackage{fullpage}
\usepackage{enumitem}
\usepackage{verbatim}
%\setlength{\textheight}{26cm} \setlength{\textwidth}{18cm}
%\setlength{\topmargin}{-0.3cm} 
%\setlength{\oddsidemargin}{-1cm}
\pagestyle{empty}
%\setlength{\parindent}{0cm}
\newcommand{\heading}[1]{\subsection*{#1}\addcontentsline{toc}{section}{#1}}
\newcommand{\subheading}[1]{\subsubsection*{#1}\addcontentsline{toc}{subsection}{#1}}
%\newcommand{\annotate}[1]{\marginpar{#1}}
\newcommand{\annotate}[1]{}

%\renewcommand{\emph}[1]{{\bf #1}}
%\newcommand\litem[1]{\item{\bfseries#1. :}}
\begin{document}

% ------ Address --------------------------------------------------------
\begin{center} \section*{\LARGE James Edwards}

\noindent
Department of Mathematics and Statistics,
%Fylde College,
Lancaster University,
Lancaster, LA1 4YF,
UK\\
T: 07821 805836\hspace{2cm} E: \url{j.edwards4@lancaster.ac.uk}         
\end{center}
\subsection*{Employment}
\begin{description}[itemsep=0.05cm,leftmargin=2.8cm,font=\bfseries,
style=multiline]
\item[2017-2019:] Senior Research Associate, Lancaster University.
\end{description}

\subsection*{Academic Qualifications}
\begin{description}[itemsep=0.05cm,leftmargin=2.8cm,font=\bfseries,
style=multiline]
\item[2012-2016:] Ph.D. in Statistics and Operational Research. 
Lancaster University.
\item[2011-2012:] M.Res. in Statistics and Operational Research,
with Distinction. Lancaster University.
\item[2010-2011:] M.Sc. in Operational Research and Management Science, with
Distinction. Lancaster University.
\item[2000-2008:] B.Sc. Hons Open, 1st class. The Open University (studied part
time due to illness).
\end{description}
%\vspace{-0.3cm}

\heading{Research}
My research is concerned with Bayesian sequential decision problems, a primary
example of which is the multi-armed bandit problem. These problems are concerned
with choosing actions over time where there is some information that is
important to each decision that is known only with uncertainty. By observing the
results of our actions we can learn more about these uncertainties and therefore
make better decisions in the future. A classic difficulty of these problems
is how to trade-off exploitation of current knowledge against exploring to gain
information.

I have contributed to this area by investigating and developing computationally
efficient heuristic methods which are important for the realistic models
required for applications. This included work showing robustness advantages of
an index heuristic over a similar non-index method even though the latter
incorporates more information. As part of this work I developed and made
available code, written in R, to enable researchers and
practitioners to calculate Gittins indices in common problem settings.

In research supported by Google, I have worked on the problem of selecting
multiple website elements (e.g. adverts or news stories) and the problem of redundancy
between similar elements. I developed new models for user actions which
incorporated correlation between elements, a Bayesian model for feedback and
learning, and fast algorithms to select elements to maximise user clicks over
time.

% Thirdly, the research explored the problem of choosing multiple website elements
% (e.g. adverts, news stories, movies) to present to a user, a problem which is
% common in many web-based businesses. If these elements are chosen individually
% then there is the danger that the elements are too similar to each other and so
% many are redundant. A new model and solution method was developed for such
% problems which provided a way of selecting a diverse set of elements so that at
% least one of these elements would be of interest to the user.


% On the theoretical side my contribution to this area has been the investigation
% and development of computationally efficient heuristic methods for the
% multi-armed bandit problem with a focus on behaviour where standard simplifying
% assumptions are relaxed. Heuristics are important because in these settings,
% which are necessary for realistic models for applications, exact solutions based
% on Dynamic Programming are rarely practical. On the application side I worked on
% a year long project with Google on the problem of selecting multiple website
% elements (e.g. adverts or news stories). Treating these elements as independent
% is unrealistic as similar elements create redundancy in the set. I developed
% new models for user actions which incorporated correlation between elements, a
% Bayesian model for feedback and learning, and fast algorithms to select elements
% to maximise user clicks over time.
\heading{Publications}
\begin{description}
\item Edwards, J. and Leslie, D. (2018) Selecting Multiple Web Adverts - a Contextual Multi-armed Bandit with State Uncertaintly. \emph{Journal of the Operational Research Society}, in press.
\item Edwards, J. and Leslie, D. (2018) Diversity as a Response to User Preference Uncertainty. \emph{Statistical Data Science.}
\item Edwards, J., Fearnhead, P. and Glazebrook, K. (2017) On the
Identification and Mitigation of Weaknesses in the Knowledge Gradient Policy for
Multi-Armed Bandits. \emph{Probability in the Engineering and Informational
Sciences}.
\item Edwards, J. (2016) Exploration and exploitation in Bayes sequential
decision problems. \emph{PhD thesis,} Lancaster University.
\url{http://eprints.lancs.ac.uk/84589/}


  
\end{description}

\heading{Experience}
\begin{itemize}
  \item Graduate teaching assistant at Lancaster University for both the
  Management Science and the Mathematics and Statistics departments. Subjects
  include: Probability, R, Latex, Business Analytics, Bayesian Statistics,
  Programming for Data Science and Statistical Inference.
%   \begin{itemize}
%     \item At undergraduate level: Statistics, Probability, R, Latex and Business
%   Analytics.
%     \item At Masters level: Bayesian Statistics, Programming for Data Science
%     and Statistical Inference.
%   \end{itemize} 
  \item Running the R part of the Project Skills undergraduate course. This
  involved editing notes, writing and administrating a test, collating marks and handling student
  queries and issues.
  \item Member of the committee which organised the Understanding
  Complex Large Industrial Data (UCLID) conference. This was student led at
  every stage including funding application.
\end{itemize}
%\vspace{-0.3cm}
\heading{Extra Training and Skills}
The STOR-i Centre for Doctoral Training gave opportunities for extensive
training beyond that required for the PhD. Examples include:
\begin{itemize}
  \item I supervised an 8 week intern research project designed by myself.
  \item Industry problem solving days. Work in groups on a problem
  presented by a company or other non-academic organisation. Example companies
  include Shell, BT and DSTL.
  \item Extended training in statistics and operational
  research topics at NATCOR and APTS courses and STOR-i
  organised masterclasses given by international experts in their research area.
  \item Training in vocal and written communication from Vox coaching,
  Michael Blastland and Andrew Garratt.
  \item Giving regular research talks at a weekly forum.
  \item Active participation in student led groups such as a research computing
  group and a machine learning reading group.
%   \item An annual STOR-i conference with speakers from both academia and
%   industry.
%  \item Leadership talks from leading individuals from industry and academia. 

\end{itemize}
\vspace{-0.3cm}
\heading{Personal Interests}
I enjoy running (especially fell and trail), cycling and cooking.\\

\noindent\textbf{References available on request.}
\begin{comment}
Peter Frazier visit grant.
NPS work and visit.
EPQ
Stats summer school
Florence Nightingale
OU award
UCLID grant �6025 from STOR-i research fund in Nov 2013
Supervision of intern
PSDs
Masterclasses and training courses

Conferences:
IMA/ORS Birmingham - April 2017 - gave talk (invited)
RSS - Glasgow - Sept 2017 - gave talk (invited)
Emergent and Self-Adaptive Systems workshop (LU) Oct 2017
Multi-armed bandits workshop - Rotterdam - May 2018 (poster)
Stochmod - Lancaster - June 2018

Other talks:
Talk to DSI on bandits
Maths \& stats department seminar

Collaborative workshops:
Oslo meetup Nov 2017 (2 days?)
One day workshop at Lancaster with Simen (Oslo)

Training:
OED Research developer programme
Peer-to-peer coaching course
OED courses: 
- Introduction to REF Impact
- Research Grants- Lancaster Costing and Approval Processes
- Intellectual Property and Research Contracts
- Research funding opportunities in STEM/Health
- Pathways to Impact 
- How to Publish Strategically to Maximise Your Impact and Reach


\end{comment}
\end{document}




